\documentclass[../bericht.tex]{subfiles}

\begin{document}

  \chapter{Einleitung}

    Die Sekunde ist eine der sieben Basiseinheiten des internationalen Einheitensystems (\textit{SI-System}). Damit ist eine präzise und gleichzeitig reproduzierbare Definition notwendig. Die Sekunde ist definiert als das $9.192.631.770$-fache der Periodendauer der dem \"Ubergang zwischen den beiden Hyperfeinstrukturniveaus des Grundzustandes $6^2S_{1/2}$ von Atomen des Nuklids $\mathrm{^{133}Cs}$ entsprechenden Strahlung. Genau diese Periodendauer soll im folgenden Versuch gemessen werden.

    Weiter soll ein Termschema von C\"asium erstellt werden, welches die Hyperfeinstruktur des Grundzustandes $\mathrm{6^2S_{1/2}}$ und des angeregten Zustandes $\mathrm{6^2P_{3/2}}$ enthält.

    Für beide Aufgaben wird sowohl die Feinstruktur als auch die Hyperfeinstruktur erkl\"art und mittels der Laserspektroskopie die  \"Uberg\"ange aufgel\"ost.

\end{document}

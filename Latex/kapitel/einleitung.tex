\documentclass[../bericht.tex]{subfiles}

\begin{document}

  \chapter{Einleitung}

    Als \textit{Raman-Streuung} oder \textit{Raman-Effekt} bezeichnet man die inelastische Streuung von Photonen an Molekülen. Sie wurde 1928 von C. V. Raman nachgewiesen und nach diesem benannt. Der Effekt wird bei der \textit{Raman-Spekroskopie} ausgenutzt um Materialeigenschaften bezüglich der Struktur zu untersuchen und kann außerdem, wie im folgenden Verusch, zur Temperaturbestimmung eines Stoffes genutzt werden. Aufgrund der Aufschlussreichen Anwendung zu Stukturanalysen findet die \textit{Raman-Spektroskopie} vor Allem in der Chemie und Biologie seine Anwendung. Auch die Biomedizin, sowie der Biotechnologie und Pharmazie nutzen den \textit{Raman-Effekt} zur Chrakterisierung von synthetisierten Molekülen (z.B. Wirkstoffen in Medikamenten).

\end{document}
